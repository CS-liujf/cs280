\documentclass[12pt]{article}%
\usepackage{amsfonts}
\usepackage{fancyhdr}
\usepackage{comment}
\usepackage[a4paper, top=2.5cm, bottom=2.5cm, left=2.2cm, right=2.2cm]%
{geometry}
\usepackage{times}
\usepackage{amsmath}
\usepackage{changepage}
\usepackage{amssymb}
\usepackage{graphicx}%
\setcounter{MaxMatrixCols}{30}
\newtheorem{theorem}{Theorem}
\newtheorem{acknowledgement}[theorem]{Acknowledgement}
\newtheorem{algorithm}[theorem]{Algorithm}
\newtheorem{axiom}{Axiom}
\newtheorem{case}[theorem]{Case}
\newtheorem{claim}[theorem]{Claim}
\newtheorem{conclusion}[theorem]{Conclusion}
\newtheorem{condition}[theorem]{Condition}
\newtheorem{conjecture}[theorem]{Conjecture}
\newtheorem{corollary}[theorem]{Corollary}
\newtheorem{criterion}[theorem]{Criterion}
\newtheorem{definition}[theorem]{Definition}
\newtheorem{example}[theorem]{Example}
\newtheorem{exercise}[theorem]{Exercise}
\newtheorem{lemma}[theorem]{Lemma}
\newtheorem{notation}[theorem]{Notation}
\newtheorem{problem}[theorem]{Problem}
\newtheorem{proposition}[theorem]{Proposition}
\newtheorem{remark}[theorem]{Remark}
\newtheorem{solution}[theorem]{Solution}
\newtheorem{summary}[theorem]{Summary}
\newenvironment{proof}[1][Proof]{\textbf{#1.} }{\ \rule{0.5em}{0.5em}}

\newcommand{\Q}{\mathbb{Q}}
\newcommand{\R}{\mathbb{R}}
\newcommand{\C}{\mathbb{C}}
\newcommand{\Z}{\mathbb{Z}}

\begin{document}

\title{CS280 Spring 2024 Assignment 3 \\ Part A}
\author{RNN, LSTM and GRU}
\maketitle

\paragraph{Name:}

\paragraph{Student ID:}

\newpage


\subsubsection*{1. RNN (16 points)}
Ronnie makes a simple RNN with state dimension 1 and a step function for $f_1$, so that
$$s_t = \text{step}(w_1x_t+w_2s_{t-1}+b)$$
where \[ \text{step}(z)=\left\{
	\begin{array}{ll}
	  1,\ z>0\\
	  0,\ \text{otherwise}\\
	\end{array}
  \right. \] and the output $$y_t=s_t$$.

\begin{itemize}
	\item[(a)] Assuming $s_0 = 0$, what values of $w_1$, $w_2$ and $b$ would generate output sequence
	$$[0, 0, 0, 1, 1, 1, 1]$$
	given input sequence
	$$[0, 0, 0, 1, 0, 1, 0]$$
	\\\\\\\\\\\\\\
	% Give your answer here.



	\item[(b)] Now Ronnie wants to make their machine generate output sequence
	$$[1, 1, 1, 0, 0, 0, 1, 1]$$
	given input sequence
	$$[0, 0, 0, 1, 0, 0, 1, 0]$$
	Assuming $s_0 = 1$, provide the $s_t$ value for each possible combination of $x_t$ and $s_{t-1}$
values, according to the input and output sequences in this example.
\begin{center}
	\begin{tabular}{|c|c|c|}  
		\hline
		$\ x_t\ $& $s_{t-1}$ &$\ s_t\ $\\ \hline
		0&0& \\ \hline
		0&1& \\ \hline
		1&0& \\ \hline
		1&1& \\ \hline
		\end{tabular} 
\end{center}

\newpage
		\item[(c)] Do you think it is possible to use Ronnie’s architecture to implement the mapping from the input sequence to the output sequence in question (b)?
		If you answered \textbf{not possible}, explain why and describe a change to this architecture that
		can implement this mapping. If you answered \textbf{possible}, provide the parameters $w_1$, $w_2$, and
		$b$ required to implement this mapping.
	\end{itemize}


\newpage


\subsubsection*{2. LSTM and GRU (17 points)}

\begin{itemize}
	\setlength{\itemsep}{-1pt}
	\setlength{\parsep}{0pt}
\setlength{\parskip}{0pt}
	\item[(a)] Please describe the forward propagation process of LSTM, and analyze why LSTM can \\mitigate the vanishing and exploding gradient problems in vanilla RNN.\\
	\item[(b)] Please describe the differences between LSTM and GRU.\\ 
\end{itemize}
\newpage





\end{document}